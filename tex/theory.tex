\section{Theoretical background}


\subsection{Mathematical Overview of the Models}
\subsubsection{Logistic Regression}
The backbone of logistic regression is linear regression, i.e. finding the least-squares solution to an equation system \begin{equation}
    X\theta = b
\end{equation}
given by the normal equations \begin{equation}
    X^TX \theta = X^Tb
\end{equation}
where $X$ is the training data matrix, $\theta$ is the coefficient vector and $b$ is the training output. The parameter vector is then used in the sigmoid function: \begin{align}
    \sigma(z) &= \frac{e^{z}}{1+e^{z}}: \; \mathbb{R}\to [0,1],\\
    z &= x^T \theta,
\end{align}
where $x$ is the testing input. This gives a statistical interpretation of the input vector. In the case of a binary True/False classification, the value of the sigmoid function then determines the class.

\subsection{Input Data Modification}
By plotting the data and analyzing the .csv file, some observations were made. The different inputs were then changed accordingly:
\begin{itemize}
    \item \emph{Kept as-is}: \texttt{weekday}, \texttt{windspeed}, \texttt{visibility}, \texttt{temp}
    \item \emph{Modified}:
    \begin{itemize}
        \item \texttt{month} - split into two inputs, one cosine and one sine part. This make the new inputs linear and can follow the fluctuations of the year. The original input was discarded.
        \item \texttt{hour\_of\_day} - split into three boolean variables: \texttt{demand\_day}, \texttt{demand\_evening}, and \texttt{demand\_night}, reflecting if the time was between 08-14, 15-19 or 20-07 respectively. This was done because plotting the data showed three different plateaues of demand for the different time intervals. The original input was discarded.
        \item \texttt{snowdepth}, \texttt{precip} were transformed into booleans, reflecting if it was raining or if there was snow on the ground or not. This was done as there was no times where demand was high when it was raining or when there was snow on the ground.
    \end{itemize} 
    \item \emph{Removed}: \texttt{cloudcover}, \texttt{day\_of\_week}, \texttt{snow}, \texttt{dew}, \texttt{holiday}, \texttt{summertime}, \texttt{humidity}. These were removed due to being redundant (e.g. \texttt{summertime}), not showing a clear trend (e.g. \texttt{cloudcover}) or both (e.g. \texttt{day\_of\_week}).
\end{itemize}