\section{Plan}
\subsection{From Intro}

\begin{enumerate}[(i)]
    \item Explotre and preprocess data
    \item try some or all classification methods, which are these?
    \begin{itemize}
        \item Logistic Regression
        \item Discriminant analysis: LDA, QDA
        \item K-nearest neighbor
        \item Tree-based methods: classification trees, random forests, bagging
        \item Boositing
    \end{itemize} 
    \item Which of these are to be "put in producion"?
\end{enumerate}

\subsection{From Data analysis task}
\begin{itemize}
    \item Can any trend be seen comparing different hours, weeks, months?
    \item Is there any diffrence between weekdays and holidays?
    \item Is there any trend depending on the weather?
\end{itemize}

\subsection{From Implementation of methods}

Each group member should implement one family each, who did what shall be clear!

DNNs are encouraged to be implemented, do this if there is time. 
(DNN is not a thing a group member can claim as their family.)

Implement a naive version, let's do: $\textit{Always low\_bike\_demand}$
\subsubsection{What to do with each method}
\begin{enumerate}
    \item Implement the method (each person individually)
    \item Tune hyper-parameters, discuss how this is done (each person individually)
    \item Evaluate with for example cross-validation. Don't use $E_{k-fold}$ (what is that?) (need to do together)
    \item (optional) Think about input features, are all relevant? (together)
\end{enumerate}

Before training, unify pre-processing FOR ALL METHODS and choose ONE OR MULTIPLE metrics to evaluate the model.
(is it neccesary to have the same for all?, is it beneficial?) Examples:
\begin{itemize}
    \item accuracy
    \item f1-score
    \item recall
    \item precision
\end{itemize}

Use same test-train split for ALL MODELS