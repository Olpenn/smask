\section{Data Analysis}
    In the given data, there are some numerical and categorical features:
        \begin{itemize}
            \item \emph{Numerical}: \texttt{temp}, \texttt{dew}, \texttt{humidity}, \texttt{precip}, \texttt{snow}, \texttt{snowdepth}, \texttt{windspeed}, \texttt{cloudcover} and \texttt{visibility}.
            \item \emph{Categorical}: \texttt{hour\_of\_day}, \texttt{day\_of\_week}, \texttt{month}, \texttt{holiday}, \texttt{weekday}, \texttt{summertime}, and \texttt{increase\_stock}
        \end{itemize}

    \begin{figure}[htbp]
        \centering
        \begin{subfigure}{0.45\textwidth}
            \centering
            \includegraphics[width=\textwidth]{demand_day.pdf}
            \caption{Demand per day of week.}
            \label{fig:demand day}
        \end{subfigure}
        \hfill
        \begin{subfigure}{0.45\textwidth}
            \centering
            \includegraphics[width=\textwidth]{demand_month.pdf}
            \caption{Demand per month.}
            \label{fig:demand month}
        \end{subfigure}
        \hfill
        \begin{subfigure}{0.8\textwidth}
            \centering
            \includegraphics[width=\textwidth]{demand_hour.pdf}
            \caption{Demand per hour of day.}
            \label{fig:demand hour}
        \end{subfigure}
        \caption{Bike demand vs. day of week and month.}
        \label{fig:demand day month}
    \end{figure}
    
    \begin{figure}[htbp]
        \centering
    
        \begin{subfigure}{0.30\textwidth}
            \centering
            \includegraphics[width=\textwidth]{demand_cloudcover.pdf}
            \caption{Demand per cloudcover (percentage).}
            \label{fig:demand cloudcover}
        \end{subfigure}
        \hfill
        \begin{subfigure}{0.30\textwidth}
            \centering
            \includegraphics[width=\textwidth]{demand_snowdepth.pdf}
            \caption{Demand per snowdepth (mm).}
            \label{fig:demand snowdepth}
        \end{subfigure}
        \hfill
        \begin{subfigure}{0.30\textwidth}
            \centering
            \includegraphics[width=\textwidth]{demand_precip.pdf}
            \caption{Demand per precipitation (mm/hour).}
            \label{fig:demand precip}
        \end{subfigure}
        \hfill
        \begin{subfigure}{0.30\textwidth}
            \centering
            \includegraphics[width=\textwidth]{demand_dew.pdf}
            \caption{Demand per dew point ($^\circ$C).}
            \label{fig:demand dew}
        \end{subfigure}
        \hfill
        \begin{subfigure}{0.30\textwidth}
            \centering
            \includegraphics[width=\textwidth]{demand_temp.pdf}
            \caption{Demand per temperature ($^\circ$).}
            \label{fig:demand temp}
        \end{subfigure}
        \hfill
        \begin{subfigure}{0.30\textwidth}
            \centering
            \includegraphics[width=\textwidth]{demand_humidity.pdf}
            \caption{Demand per humidity level (percentage).} 
            \label{fig:demand humidity}
        \end{subfigure}
            \caption{Bike demand vs. various weather parameters.}
            \label{fig:demand weather}
    \end{figure}

    Few trends can be detected when analysing the time and the weather. From figure \ref{fig:demand day month}, one can see a periodic relationship for the months, where there is a higher demand during the warmer months, which loosely follows a trigonometric curve. Over the week, the demand is rather stable, with a peak on the weekend, specifically Saturdays. 
    
    Weather wise, Figure \ref{fig:demand weather} suggests that the demand is almost always low, whenever it is raining or snowing. Cloudcover did not make a big impact, which is also intuitive, as a cloudy day does not affect biking as much as snowing for example. Similarly to cloudcover, dew point does not have a clear trend, while humidity however has; when the humidity increases, the demand decreases. Furthermore, temperature has a somewhat clear impact on the trend; warmer weather means increase in demand.

    The overall trend is that about one eight of observations correspond to a high bike demand. During night, or bad weather, the demand is (intuitively) low. But during rush hour (see Figure \ref{fig:demand hour}), the demand is very high, and the  bike availability should probably be increased in order to minimize excessive CO$_2$ emissions.